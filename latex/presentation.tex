\documentclass[aspectratio=169]{beamer}

\usepackage[utf8]{inputenc} % codificacao de caracteres
\usepackage[T1]{fontenc}    % codificacao de fontes
\usepackage[brazil]{babel}  % idioma


\usetheme{Berlin}         % tema
\usecolortheme{seahorse}      % cores

\usefonttheme[onlymath]{serif} % fonte modo matematico
\usepackage{amsthm}

% Titulo
\title[\sc{Visão computacional}]{Visão Computacional - Gaussian Mixture Model}
\author[Alano Martins Pinto]{Alano Martins Pinto}
\institute{UECE - Universidade Estadual do Ceará} % opcional
\date{\today}


% Colocando numero de paginas no slide
\setbeamertemplate{footline}[frame number]

% Desativando os botoes de navegacao
\beamertemplatenavigationsymbolsempty

% Tela cheia\section{section1}

\hypersetup{pdfpagemode=FullScreen}

% Layout da pagina
\hypersetup{pdfpagelayout=SinglePage}

% Definicao de novos comandos
\providecommand{\sin}{} \renewcommand{\sin}{\hspace{2pt}\textrm{sen}}
\providecommand{\tan}{} \renewcommand{\tan}{\hspace{2pt}\textrm{tg}}
\newcommand{\R}{\mathbb{R}}

% Capa - requer o TikZ
\newcommand{\capa}{
    \begin{tikzpicture}[remember picture,overlay]
        \node at (current page.south west)
            {\begin{tikzpicture}[remember picture, overlay]
                \fill[shading=radial,top color=orange,bottom color=orange,middle color=red] (0,0) rectangle (\paperwidth,\paperheight);
            \end{tikzpicture}
          };
    \end{tikzpicture}
}


% Definicao de novos ambientes
\theoremstyle{Definition}
\newtheorem{defn}{Defini\c c\~ao}
\newtheorem{teo}[theorem]{Teorema}
\newtheorem{ex}[theorem]{Exemplo}
\newtheorem{eq}[theorem]{Equa\c c\~ao}


\usepackage{etoolbox}
\newcommand{\zerodisplayskips}{%
  \setlength{\abovedisplayskip}{0pt}%
  \setlength{\belowdisplayskip}{0pt}%
  \setlength{\abovedisplayshortskip}{0pt}%
  \setlength{\belowdisplayshortskip}{0pt}}
\appto{\normalsize}{\zerodisplayskips}
\appto{\small}{\zerodisplayskips}
\appto{\footnotesize}{\zerodisplayskips}




\begin{document}

\begin{frame}
	\titlepage
\end{frame}


\begin{frame}
	\frametitle{Tópicos}
	\begin{itemize}
		\item Descrição
		\item Expectation–maximization (EM)
		\item GMM covariance
		\item Estimativa de densidade
		\item Inferência Variacional
		\item Dirichlet Process

	\end{itemize}

\end{frame}

\begin{frame}
	\frametitle{Descrição}
	
	\begin{defn}
		É um modelo probabilístico para representar subpopulações normalmente distribuídas dentro de uma população geral.
	\end{defn}
	
	\begin{eq}
			$ p(x) = \sum_{i=1}^{K} \phi_i N(x \vert \mu_i, \sigma_i) $ \\
			$ N(x \vert \mu_i, \sigma_i) = 1 / { \frac{1}{\sigma_i \sqrt{2\pi}}} exp(- \frac{(x - \mu_i)^2}{2\sigma_i^2}) $ \\
			$ \sum_{i=1}^{K} \phi_i = 1 $
	\end{eq}
	
	
\end{frame}

\begin{frame}
	\frametitle{Expectation–maximization (EM)}
	
	\begin{defn}
		É um método iterativo para encontrar estimativas de máxima verossimilhança ou máximo a posteriori (MAP) de parâmetros em modelos estatísticos, onde o modelo depende de variáveis latentes não observada.
	\end{defn}
	
	\begin{eq}
			$ p(X \vert \lambda^{new}) \geq p(X \vert \lambda^{old}) $ 
	\end{eq}
	
	
\end{frame}

\begin{frame}
	\frametitle{Expectation (E)}
	
	\begin{defn}
	 Uma função para a expectativa da probabilidade logarítmica avaliada usando a estimativa atual para os parâmetros
	\end{defn}
	
	\begin{eq}
			$ \gamma_{i,k} = \frac{ \phi_k N(x_i \vert \mu_k,\sigma_k)}{    \sum^k_{j=1}  \phi_j N(x_i \vert \mu_j,\sigma_j) } $ \\
			$ \gamma_{i,k} = p(C_k \vert x_i,\phi,\mu,\sigma) $
	\end{eq}
	
	
\end{frame}


\begin{frame}
	\frametitle{Maximization (M)}
	
	\begin{defn}
		Calcula os parâmetros que maximizam a probabilidade encontrada no passo E.
	\end{defn}
	
	\begin{eq}
			$ \phi_k = \sum^N_{i=1} \frac{\gamma_{i,k}}{N} $ \\
			$ \mu_k = \frac{\sum^N_{i=1} \gamma_{i,k} x_i}{\sum^N_{i=1} \gamma_{i,k}} $ \\
			$ \sigma_k = \frac{\sum^N_{i=1} \gamma_{i,k} (x_i - \mu_k)^2}{\sum^N_{i=1} \gamma_{i,k}} $
	\end{eq}
	
	
\end{frame}


\AtEndDocument{\usebeamertemplate{endpage}}

\end{document}